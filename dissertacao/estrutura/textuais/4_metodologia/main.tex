%METODOLOGIA------------------------------------------------------------------------------------------------------
\chapter{METODOLOGIA}
\label{chap:metodologia}

A título de atingir-se os objetivos apresentados no \chaptername{} \ref*{chap:objetivos} - \nameref{chap:objetivos}, definiu-se uma estratégia de trabalho segmentada em fases, as quais sucederão umas às outras sequenciamente. Dessa forma, a estrutura definidas é a colocada abaixo:

\begin{enumerate}
    \item \textbf{Levantamento de Bibiografia --} etapa na qual serão utilizados mecanismos para identificação de literatura pertinente à área de trabalho escolhida. Para tal, considera-se o uso da ferramenta \textit{Google Scholar}, dada sua funcionalidade em retornar em seus resultados não apenas artigos e periódicos, mas também livros. Para a escolha dos trabalhos basilares julga-se possível usar como critério de filtro o volume de trabalhos de pesquisa que citam cada publicação -- considera-se aqui que, quanto maior o número de citações em que um dado trabalho figura, mais relevante o mesmo será para a área. Inicialmente, a pesquisa será focada em termos como ``\textit{Natural Language Processing}'', ``\textit{Natural Language Understanding}'', ``\textit{chatbots}'' e ``\textit{Machine Learning}'', assim como suas siglas, embora o número de termos possa ser naturalmente expandido conforme se revele necessário.
    \item \textbf{Condensação do Material --} dado que o estágio anterior pode produzir um grande montante de informações, será importante reservar tempo para analisá-las e descartar o que não se adequar ao contexto trabalhado. A príncipio, dado que anteriormente já se aplicou um filtro baseado em relavância dos trabalhos, o descarte ou seleção dos materiais deve ocorrer principalmente a partir da leitura de seus resumos ou mesmo introdução.
    \item \textbf{Leitura e Fichamento do Material; Redação da Fundamentação Teórica --} nesta fase concentra-se o esforço em se consumir todo o material anteriormente elencado, executando-se uma leitura mais criteriosa e realizando-se o processo de fichamento. Este trabalho demonstra-se necessário pois possibilita uma construção mais apurada da bibliografia a ser empregada na redação da Fundamentação Teórica, a qual será executada também na corrente etapa.
    \item \textbf{Aquisição da Base de Dados --} com a primeira parte teórica do trabalho concluído, planeja-se sua execução prática, que será iniciada pela construção de uma base de dados apropriadamente anotada para o contexto do trabalho sugerido; a especificação da base de dados (definição do número de amostras, classes presentes etc) também ocorrerá neste ciclo.
    \item \textbf{Desenvolvimento da Aplicação de Chatbot --} há a possibilidade de que este estágio ocorra paralelamente ao anterior, porém, para melhor mapeamento do trabalho, optou-se, num primeiro momento, por sua  segregação; aqui será o momento reservado à construção do robo de conversação, o qual é o cerne do presente projeto (escolha da plataforma para construção do \textit{chatbot}, definição de entidades, intenções e enunciados, programação do sistema, etc).
    \item \textbf{Testes e Observação de Resultados --} finalizada a construção da solução planejada, será necessário a execução de testes para verificar sua assertividade; o resultado de tais testes deverá ser compilado e registrado enquanto resultados obtidos, atestando o êxito do projeto ou a necessidade de apurar o mesmo. 
    \item \textbf{Revisão do Texto/Redação Final --} embora não esteja explícito, o processo de redação acompanhará todas as fazes anteriores, coexistindo com as mesmas de forma indistinguível; todavia, optou-se por reservar um período do planejamento para realizar a revisão do material produzido, o qual também será usado para a construção das considerações sobre o que foi desenvolvido e possiblidades de trabalhos futuros.
\end{enumerate}