% SUBSEÇÃO 1: FUNDAMENDAÇÃO CONCEITO B----------------------------------------------------------------------------
\section{Aprendizado de Máquina}
\label{sec:ml}

\citeonline[tradução nossa]{alpaydin2020introduction} introduz a ideia dos \textbf{algorítimos} como o núcleo da computação, o qual pode ser definido como ``uma sequência de intuções que deve ser executada para transformar a entrada em uma saída.''\footnote{``\textit{An algorithm is a sequence of instructions that should be carried out to transform the input to an output.}''}. O autor aponta, ainda, que, desde o desenvolvimento do primeiro computador, intenta-se o desenvolvimento de algorítimos para uma grande natureza de tarefas, tornando-os indispensáveis ao meio de vida adotado pelo homem atualmente; entretanto, mesmo que peça fundamental ao cotidiano humano, ainda há, de fato, uma infinidade de tarefas que são executadas facilmente pelos seres humanos para as quais o desenvolvimento de um algorítimo parece virtualmente incomcebível -- e para suprir tal carência é que surge o conceito de \textbf{Aprendizado de Máquina } (do inglês \textit{Machine Learning}, ou apenas \textbf{ML}) \cite{alpaydin2020introduction}.

\citeonline[tradução nossa]{michalski2013machine} declaram que ``o aprendizado é um fenômeno multifacetado. Os processos de aprendizagem incluem a aquisição de novos conhecimentos declarativos, o desenvolvimento de habilidades motoras e cognitivas através de instrução ou prática, a organização de novos conhecimentos em representações gerais e efetivas e a descoberta de novos fatos e teorias através da observação e experimentação.''\footnote{``\textit{Learning is a many-faceted phenomenon. Learning processes include the acquisition of new declarative knowledge, the development of motor and cognitive skills through instruction or practice, the organization of new knowledge into general, effective representations, and the discovery of new facts and theories through observation and experimentation}''}. Tanto \citeonline{alpaydin2020introduction} quanto \citeonline{michalski2013machine} apontam que o principal objetivo da disciplina de Aprendizado de Máquina é oferecer aos sistemas computacionais a capacidade de aprender tal qual observado no ser humano; tal afirmação é reforçada por \citeonline{el2015machine} ao escreverem que tal área de trabalho enquadra-se numa categoria de algorítmos que são capazes de emular alguns aspectos da inteligência humana. Condensando-se os conceitos anteriores, pode-se afirmar que o Aprendizado de Máquina constitui-se de capacitar a máquina da habilidade de aprender, sem que um ser humano a tenha programado explicitamente \cite{samuel1988some}.
