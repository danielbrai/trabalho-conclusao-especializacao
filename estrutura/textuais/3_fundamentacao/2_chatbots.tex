% SUBSEÇÃO 1: FUNDAMENDAÇÃO CONCEITO B----------------------------------------------------------------------------
\section{Chatbots}
\label{sec:chatbots}

Tal qual proposto por \citeonline{wezel2020m}, um \textit{chatbot} -- uma contração para o termo \textbf{robo de conversação} (do inglês, \textit{chatting robot}) \cite{lokman2018modern}, é uma aplicação baseada em diálogo, projetada para atender a uma finalidade específica e bem definida. Intentando uma expansão deste conceito, pode-se seguir a linha proposta por \citeonline{mctear2020conversational}, o qual pontua que tais sistemas são desenvolvidos para suportar interações com humanos estabelecidas de forma escrita, por fala ou mesmo por ambas as interfaces citadas; tais interações podem ser classificadas em \textbf{diálogos orientados à tarefas}, no qual o ser humano e o sistema estabelecem uma comunicação visando a completude de uma atividade qualquer, e em \textbf{diálogos não orientados a tarefas}, no qual a interação ocorre sem qualquer finalidade pré estabelecida, sendo o objetivo do sistema proporcionar àqueles que com ele interagem uma experiência póxima à comunicação rotineira obervada entre seres humanos. Esta interação é possibilitada graças à existência e à aplicação das técnicas de Processamento de Linguagem Natural \cite{lokman2018modern}.

Pode-se traçar um paralelo entre o surgimento das primeiras aplicações visando esta finalidade -- em meados da década de 1960, com o desenvolvimento da aplicação ELIZA, desenvolvida pelo MIT -- e a evolução dos estudos relacionados à disciplina de NLP \cite{lokman2018modern,allen1988natural}. Todavia, embora tenha permeado o campo computacional desde o seu surgimento, nota-se que o tema ganhou maior relevância recentemente. \citeonline{lokman2018modern} pontuam que tal eminência deve-se, principalmente, ao fato de os dispositivos celulares terem sofrido uma alteração em seu modo de operação: hoje, a troca de mensagens curtas de texto, que representa uma comunicação mais ágil e enxuta, tem-se mostrado o principal uso dos aparelhos, em detrimento à operação por voz, a qual caracteriza um meio de comunicação longo; outro fator que justificaria o recente enfoque ao tópico seria a ``corrida''  disputada pelas grandes corporações na busca de soluções no segmento de assistentes pessoais virtuais (Amazon Alexa, Google Assistant and Apple Siri) \cite{lokman2018modern}.