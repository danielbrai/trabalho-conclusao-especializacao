% SUBSEÇÃO 1: FUNDAMENDAÇÃO CONCEITO A----------------------------------------------------------------------------
\section{Processamento de Linguagem Natural (PLN)}
\label{sec:npl}

O Processamento de Linguagem Natural (PLN, ou NLP - do inglês \textit{Natural Language Processing}) pode ser definido, conforme proposto por \citeonline{cambria2014jumping}, como ``um conjunto de técnicas computacionais para representação e análise automática da linguagem humana". Neste contexto, sendo a linguagem a principal ferramenta pela qual seres humanos estabelecem seu padrão comportamental para a vida em sociedade e, consequentemente, a forma mais natural de interação da qual rotineiramente lançam mão em seu cotidiano \cite{allen1988natural}, é compreensível que a construção de soluções computacionais que explorem tal interface -- como interpretá-la, extrair informações e mesmo compreendê-la -- tenha permeado tantos trabalhos e pesquisas ao longo das décadas, desde a primeira proposição abordando o tema, ainda na década de 1950 \cite{cambria2014jumping}. Uma das grandes motivações para explorar esta disciplina advém do fato de que o conhecimento humano está registrado majoritariamente de forma linguística (mídias audiovisuais como um todo - livros, vídeos, conteúdos de áudio e afins), logo modelos computacionais que consigam transpor a barreira da linguagem humana podem acessar -- e entender -- a toda esta informação, processando-a e tornando o processo de consumo a qualquer um que o deseje (ou necessite) fazê-lo mais simples e menos moroso, implicando em sistemas mais flexíveis e inteligentes que aqueles atualmente disponíveis \cite{allen1988natural}.

Dada a complexidade inerente à linguagem humana, o Processamento de Linguagem Natural exige uma capacidade simbólica de alto nível por parte da máquina que o está operando, a qual é naturalmente observada nos seres humanos -- revelando-se, de certa forma, uma capacidade trivial possuída pelo homem, dado serem estes os responsáveis pelo desenvolvimento da língua enquanto ferramenta comunicativa; isso decorre do fato de que cada palavra carrega em si uma intrincada relação semântica, permeada por diversos conceitos, envolvendo episódios relevantes e mesmo experiências particulares para os envolvidos no processo de comunicação. \cite{cambria2014jumping}. Para habilitar uma aplicação baseada em NLP é necessário dotá-la de um considerável conhecimento da estrutura do idioma em si na qual aquele sistema foi construído (isto é, inglês, português, espanhol, mandarim etc), o que envolve conhecer os vocábulos pertencentes àquele idioma, como as palavras se interconectam para gerar sentenças e como tais vocábulos criam sentido, divergindo de seu significado literal, a depender do contexto no qual encontram-se inseridos (denotação e conotação); por fim, é preciso que a aplicação também seja apresentado ao campo lexical encerrado pelo domínio de negócio que se busca abordar \cite{allen1988natural}.

Dado o exposto, o desenvolvimento da disciplina de NLP baseia-se, fundamentalmente, em três pilares - \textbf{Sintaxe}, o qual especifica como os símbolos significativos para a língua são agrupados logicamente; \textbf{Semântica}, responsável por definir como as expressões são formadas e qual o seu suposto significado; e \textbf{Contexto}, que encerra os mecânismos que possibilitam estabelecer correlações entre diferentes semânticas e permite a desambiguação da informação consumida. Destaca-se o fato de que os trabalhos iniciais na área, desenvolvidos ainda na década de 1950, foram construídos abordando fundamentalmente os mecânismos para o processamento da estrutura formal da língua em si (Sintaxe), justificando-se por ser (1) uma etapa necessária ao avanço para abordar os demais pilares (Semântica e Contexto) e (2) por possuir uma aplicação mais direta e imediata nas técnicas de aprendizagem de máquina \cite{cambria2014jumping}.

\input{estrutura/textuais/3_fundamentacao/1.1_figure_1.tex}

Para \citeonline{allen1988natural}, além da visão sistêmica demonstrada anteriormente, há ainda um agrupamento mais específico das áreas de conhecimento relevantes ao entendimento da linguagem; para o autor, para aplicações centradas numa interação escrita, pode-se elencar como relevantes ao sistema os seguintes conhecimentos: (1) \textbf{Conhecimento Morfológico} (conhecer como as palavras são construídas no idioma ao qual o sistema busca atender - a exemplo, as palavras ferro, ferrugem e ferradura derivam todas do mesmo radical, ferr); (2) \textbf{Conhecimento Sintático} (como as palavras podem ser agrupadas, e qual o papel de cada vocábulo na estrutura construída); (3) \textbf{Conhecimento Semântico} (implica no conhecimento do significado das palavras e como estes podem ser empregadas para a construção de sentenças com significados que possuam sentido e significado, independente do contexto no qual foram empregados); (4) \textbf{Conhecimento Pragmático} (como sentenças empregadas em contextos diferentes podem assumir significados igualmente diferentes); (5) \textbf{Conhecimento do Diálogo} (como sentenças afetam umas às outras, impactando na interpretação do discurso conduzido - principalmente em relação ao emprego de pronomes e ao fluxo do diálogo ao longo do tempo); (6) \textbf{Conhecimento de Domínio} (compete ao entendimento do contexto no qual os usuários daquele domínio empregam cada vocábulo). Para aplicações cujo meio de interação com o usuário seja a voz, há a necessidade de um campo de conhecimento adicional, o \textbf{Conhecimento Fonético e Fonológico} (que se relaciona a forma como cada vocábulo relaciona-se com os sons que os representam). O relacionamento entre as áreas apresentadas pode ser melhor observado na representação vista na figura \ref{fig:M1}.

