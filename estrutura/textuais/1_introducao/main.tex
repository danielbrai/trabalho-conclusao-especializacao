% INTRODUÇÃO-------------------------------------------------------------------

\chapter{INTRODUÇÃO}
\label{chap:introducao}
O surgimento da Web 2.0 trouxe consigo um novo tipo de aplicação, centrada primariamente ao redor de usuários (ao contrário da Web, a qual é organizada baseada em conteúdo), conhecidas como redes sociais online (do inglês \textit{online social network}, ou apenas OSN), as quais definem e gerenciam relações entre seus usuários. Consideradas uma extensão da própria Web, tais redes tornaram-se um dos mais populares serviços disponíveis na Internet, permitindo uma rica interação entre seus utilizadores, os quais podem publicar uma variedade de tipos de informação, tais qual texto, fotos e vídeos, além de comentá-las, compartilhá-las com outros usuários ou ainda executar diversas outras operações de acordo com o implementado por cada provedor \cite{pallis2011online,mislove2007measurement}. Conforme posto por \citeonline{xia2021tweet}, um dos prováveis motivos para a rápida popularização das OSNs seja o fato de tais sistemas servirem como ambiente livre para que indivíduos compartilhem suas opiniões e como canal para que se façam ser ouvidos -- e notados, o que é evidenciado através do constante crescimento de usuários das plataformas existentes nos últimos anos, além do surgimento de novos serviços de semelhante natureza \cite{pallis2011online}. 
\par
Há de se observar que, em decorrência da natureza própria apresentada pelas OSNs, tais serviços impactaram, impactam e -- provavelmente -- continuarão a impactar as relações humanas ainda por algum tempo, sejam estas relações de natureza pessoal ou comercial, assim como a forma que organiza-se, armazena-se e compartilha-se informação e conhecimento \cite{mislove2007measurement}. Quando o discurso \textit{online} é considerado, pode-se encarar as OSNs como um termômetro do que seus usuários pensam ou sentem em relação a um dado tema ou assunto; tal indicativo pode, inclusive, ser extendido aos períodos de eleições -- postagens realizadas no intervalo de tempo que antecede ao pleito podem dizer muito sobre o que um usuário -- ou grupo de usuários -- considera sobre uma ideologia, partido político ou mesmo candidato, a depender do tipo de postagem realizada e da carga de sentimento contida  nela \cite{shevtsov2020analysis}. \citeonline{chaudhry2021sentiment} aponta que uma análise sentimental empregada em postagens realizadas em redes sociais, como Twitter, Instagram ou Facebook (dentre outras), é capaz de demonstrar como a população -- ao menos a parte socialmente ativa dentro de tais plataformas -- posiciona-se em relação ao pleito e aos candidatos que o disputam, dado que são, atualmente, as principais vias utilizadas para expressar sentimentos, através da discussão de eventos políticos, acontecimentos globais e compartilhamento de notícias.
\par
Ainda que demonstrem um enorme potencial enquanto ferramenta de análise e entendimento de resultados de pleitos eleitorais, não se pode ignorar as limitações observadas quando emprega-se as OSNs para semelhante finalidade. Primeiramente -- e talvez a mais evidente -- relacione-se à amostragem que as redes sociais digitais oferecem: ao contrário do processo habitual de separação de amostras, tradicionalmente conduzido dentro da disciplina de estatística, nem toda a população está devidamente representada dentro do universo das redes sociais; esta proposição implica que, como nem todos os votantes fazem uso de tal tipo de comunidade -- e que mesmo dentre aqueles que fazem há muitos que restringem o conteúdo compartilhado através das políticas de privacidade oferecidos por cada plataforma, é impossível que tenhamos uma fiel representação da realidade. Um segundo ponto relaciona-se ao conteúdo das publicações, dado que, por conta das limitações encontradas nas ferramentas atualmente disponíveis, a análise invariavelmente sofrerá algum impacto negativo, podendo ser exemplificado pela detecção de sarcasmo -- um sentimento amplamente encontrado na dinâmica observada dentro das redes sociais -- e que não é percebido facilmente \cite{chaudhry2021sentiment} com o uso das técnicas de Processamento de Linguagem Natural (PNL, ou NLP conforme o original em inglês \textit{Natural Language Processing}). Ainda em relação aos desafios observados no uso e na análise das OSNs, \citeonline{lovera2021sentiment} argumenta que, enquanto um ambiente inteiramente informal, o emprego da língua foge quase que completamente às normas cultas do idioma em questão, qualquer que o seja este, imperando o uso de gírias, abreviações e expressões com significados próprios para nichos específicos, limitanto o uso das ferramentas de NLP em sua forma original, sem maiores ajustes ou configurações para a dinâmica das redes. Por fim, há, ainda, o isolamento existente dentro da esfera digital, as chamadas ``câmaras de eco": embora, de fato, a Internet apresente-se como uma poderosa ferramenta de disseminação de ideias e de compartilhamento de informação, promovendo a discussão pública, nota-se o quão fragmentada as OSNs podem se mostrar, atraindo e aglutinando os que, possuindo pensamentos e ideais semelhantes, isolam-se daqueles que enxergam como inimigos (por possuírem opiniões opostas) em comunidades isoladas e segragadas do restante da rede. Em tais comunidades há pouco -- quando não nenhum -- debate confrontando pensamentos divergentes entre si, enquanto abunda a replicação de ideias e opiniões comuns àqueles que delas participam, as quais circulam repetidamente e as reforçam ainda mais a identidade daquele grupo isolado \cite{takikawa2017political}.
\par
Amplamente abordado em âmbito acadêmico \cite{shevtsov2020analysis,zahrah2022comparison,chaudhry2021sentiment,takikawa2017political,xia2021tweet}, a utilização das redes sociais digitais como ferramenta de análise e predição para resultados de eleições ainda posiciona-se como estratégia válida a medida que novas técnicas para processamento de texto e afins são desenvolvidas. Ademais, a aplicação de técnicas de explicabilidade podem dotar tais análises de maior robustez, ao forncer indícios de como os modelos aplicados funcionam internamente; com este entendimento, é possível refinar tais modelos, evitando viéses e diminuindo ruídos, previnindo e retificando conclusões incorretas ao depurar o modelo aplicado. É neste contexto que o LIME (\textit{Local Interpretable Model-agnostic Explanattions}) é inserido, explicando como um dado ponto classificado de uma forma e não de outra, sendo alocado a numa classe específica, permitindo o referido entendimento do modelo aplicado, indepentente de qual técnica tenha sido aplicado para o trabalho de predição e/ou classificação \cite{confalonieri2021historical,lovera2021sentiment}, sendo a presente proposta centrada em aliar-se um modelo de classificação ao LIME, posicionando-o ao final do processo de análise de sentimentos da rede social Twitter.
A escolha do Twitter enquanto OSN para estudo deu-se pela importância que a rede assumiu enquanto ferramenta de difusão de informação e de mobilização popular para diversos movimentos políticos de impacto global, como a Primavera Árabe (2010), o OWS (\textit{Occupy Wall Street}, 2011), os protestos no Parque Gezi (2013) \cite{takikawa2017political} ou mesmo o movimento Passe Livre (2013), ao longo da última década, além de a própria rede posicionar-se como um palco para o debate público, como um lugar livre para esta finalidade.